\documentclass{homework}
\usepackage{lipsum}
\usepackage{cancel}
\usepackage{amsthm}
\usepackage{cleveref}
\usepackage{upgreek}
\usepackage[framed]{mcode}
\usepackage{mathrsfs}
\usepackage{tikz}
\usepackage{units}
\usetikzlibrary{matrix}
\newtheorem{lemma}{Lemma}
\DeclareMathOperator*{\argmin}{arg\,min}

\title{Kevin Joyce}
\course{Math 514 - Inverse Problems - Homework 5}
\author{Kevin Joyce}
\docdate{\today}
\begin{document} 
\newcommand{\figref}[1]{\figurename~\ref{#1}}
\renewcommand{\bar}{\overline}
\renewcommand{\hat}{\widehat}
\renewcommand{\SS}{\mathcal S}
\newcommand{\HH}{\mathscr H}
\newcommand{\mom}{\widetilde}
\newcommand{\mle}{\widehat \Uptheta}
\newcommand{\eps}{\varepsilon}
\newcommand{\todist}{\stackrel{D}\longrightarrow}
\newcommand{\toprob}{\stackrel{p}\longrightarrow}
\newcommand{\TTheta}{\overline{\underline \Theta} }
\newcommand{\del}{\partial}
\newcommand{\approxsim}{\overset{\cdotp}{\underset{\cdotp}{\sim}}}

Codes for each problem are available at \url{https://github.com/kjoyce/inverse_problems/tree/master/homework04/codes}

\begin{longproblem}

  \subproblem{Derive the formulas for the UPRE analogous to (3.18) for GCV.  Add lines of code to \texttt{Deblur2dPeriodic.m} so that it implements UPRE.}

  \subproblem{Derive the formulas for the DP analogous to (3.18) for GCV.  Add lines of code to \texttt{Deblur2dPeriodic.m} so that it implements DP regularization parameter selection methods.}

\end{longproblem}

\begin{longproblem}
Modify \texttt{Deblur2DataDriven.m} so that the truncated Landweber iteration, introduced in Chapter 2, is used for solving the deblurring problem.  Use the DP stopping rule, i.e. choose the first $k$ such that $\|\vect{Ax_k} - \vect b\|^2 \le n^2\sigma^2.$
\end{longproblem}

\end{document} 
